% ****************************************************************************************
% ************************        PRACTICA 1                  ****************************
% ****************************************************************************************


% =======================================================
% =======         HEADER FOR DOCUMENT        ============
% =======================================================
    
    % *********   HEADERS AND FOOTERS ********
    \def\ProjectAuthorLink{https://github.com/SoyOscarRH}           %Just to keep it in line
    \def\ProjectNameLink{\ProjectAuthorLink/Proyect}                %Link to Proyect

    % *********   DOCUMENT ITSELF   **************
    \documentclass[12pt, fleqn]{report}                             %Type of docuemtn and size of font and left eq
    \usepackage[spanish]{babel}                                     %Please use spanish
    \usepackage[utf8]{inputenc}                                     %Please use spanish - UFT
    \usepackage[margin = 1.2in]{geometry}                           %Margins and Geometry pacakge
    \usepackage{ifthen}                                             %Allow simple programming
    \usepackage{hyperref}                                           %Create MetaData for a PDF and LINKS!
    \usepackage{pdfpages}                                           %Create MetaData for a PDF and LINKS!
    \hypersetup{pageanchor = false}                                 %Solve 'double page 1' warnings in build
    \setlength{\parindent}{0pt}                                     %Eliminate ugly indentation
    \author{Oscar Rosas, Laura Andres, Alan Ontiveros}              %Who I am

    % *********   LANGUAJE    *****************
    \usepackage[T1]{fontenc}                                        %Please use spanish
    \usepackage{textcmds}                                           %Allow us to use quoutes
    \usepackage{changepage}                                         %Allow us to use identate paragraphs
    \usepackage{anyfontsize}                                        %All the sizes

    % *********   MATH AND HIS STYLE  *********
    \usepackage{ntheorem, amsmath, amssymb, amsfonts}               %All fucking math, I want all!
    \usepackage{mathrsfs, mathtools, empheq}                        %All fucking math, I want all!
    \usepackage{cancel}                                             %Negate symbol
    \usepackage{centernot}                                          %Allow me to negate a symbol
    \decimalpoint                                                   %Use decimal point

    % *********   GRAPHICS AND IMAGES *********
    \usepackage{graphicx}                                           %Allow to create graphics
    \usepackage{float}                                              %For images
    \usepackage{wrapfig}                                            %Allow to create images
    \graphicspath{ {Graphics/} }                                    %Where are the images :D

    % *********   LISTS AND TABLES ***********
    \usepackage{listings, listingsutf8}                             %We will be using code here
    \usepackage[inline]{enumitem}                                   %We will need to enumarate
    \usepackage{tasks}                                              %Horizontal lists
    \usepackage{longtable}                                          %Lets make tables awesome
    \usepackage{booktabs}                                           %Lets make tables awesome
    \usepackage{tabularx}                                           %Lets make tables awesome
    \usepackage{multirow}                                           %Lets make tables awesome
    \usepackage{multicol}                                           %Create multicolumns

    % *********   HEADERS AND FOOTERS ********
    \usepackage{fancyhdr}                                           %Lets make awesome headers/footers
    \pagestyle{fancy}                                               %Lets make awesome headers/footers
    \setlength{\headheight}{16pt}                                   %Top line
    \setlength{\parskip}{0.5em}                                     %Top line
    \renewcommand{\footrulewidth}{0.5pt}                            %Bottom line

    \lhead{                                                         %Left Header
        \hyperlink{section.\arabic{section}}                        %Make a link to the current chapter
        {\normalsize{\textsc{\nouppercase{\leftmark}}}}             %And fot it put the name
    }

    \rhead{                                                         %Right Header
        \hyperlink{section.\arabic{section}.\arabic{subsection}}    %Make a link to the current chapter
            {\footnotesize{\textsc{\nouppercase{\rightmark}}}}      %And fot it put the name
    }
    \rfoot{\textsc{\small{\hyperref[sec:Index]{Ve al Índice}}}}     %This will always be a footer  

    \fancyfoot[L]{                                                  %Algoritm for a changing footer
        \ifthenelse{\isodd{\value{page}}}                           %IF ODD PAGE:
            {\href{https://compilandoconocimiento.com/nosotros/}    %DO THIS:
                {\footnotesize                                      %Send the page
                    {\textsc{Oscar Rosas, Laura Andres, Alan Ontiveros}}}}                 %Send the page
            {\href{https://compilandoconocimiento.com}              %ELSE DO THIS: 
                {\footnotesize                                      %Send the author
                    {\textsc{Practica 1}}}}            %Send the author
    }
    
    
    
% =======================================================
% ===================   COMMANDS    =====================
% =======================================================

    % =========================================
    % =======   NEW ENVIRONMENTS   ============
    % =========================================
    \newenvironment{Indentation}[1][0.75em]                         %Use: \begin{Inde...}[Num]...\end{Inde...}
        {\begin{adjustwidth}{#1}{}}                                 %If you dont put nothing i will use 0.75 em
        {\end{adjustwidth}}                                         %This indentate a paragraph
    \newenvironment{SmallIndentation}[1][0.75em]                    %Use: The same that we upper one, just 
        {\begin{adjustwidth}{#1}{}\begin{footnotesize}}             %footnotesize size of letter by default
        {\end{footnotesize}\end{adjustwidth}}                       %that's it

    \newenvironment{MultiLineEquation}[1]                           %Use: To create MultiLine equations
        {\begin{equation}\begin{alignedat}{#1}}                     %Use: \begin{Multi..}{Num. de Columnas}
        {\end{alignedat}\end{equation}}                             %And.. that's it!
    \newenvironment{MultiLineEquation*}[1]                          %Use: To create MultiLine equations
        {\begin{equation*}\begin{alignedat}{#1}}                    %Use: \begin{Multi..}{Num. de Columnas}
        {\end{alignedat}\end{equation*}}                            %And.. that's it!
    

    % =========================================
    % == GENERAL TEXT & SYMBOLS ENVIRONMENTS ==
    % =========================================
    
    % =====  TEXT  ======================
    \newcommand \Quote {\qq}                                        %Use: \Quote to use quotes
    \newcommand \Over {\overline}                                   %Use: \Bar to use just for short
    \newcommand \ForceNewLine {$\Space$\\}                          %Use it in theorems for example

    % =====  SPACES  ====================
    \DeclareMathOperator \Space {\quad}                             %Use: \Space for a cool mega space
    \DeclareMathOperator \MegaSpace {\quad \quad}                   %Use: \MegaSpace for a cool mega mega space
    \DeclareMathOperator \MiniSpace {\;}                            %Use: \Space for a cool mini space
    
    % =====  MATH TEXT  =================
    \newcommand \Such {\MiniSpace | \MiniSpace}                     %Use: \Such like in sets
    \newcommand \Also {\MiniSpace \text{y} \MiniSpace}              %Use: \Also so it's look cool
    \newcommand \Remember[1]{\Space\text{\scriptsize{#1}}}          %Use: \Remember so it's look cool
    
    % =====  THEOREMS  ==================
    \newtheorem{Theorem}{Teorema}[section]                          %Use: \begin{Theorem}[Name]\label{Nombre}...
    \newtheorem{Corollary}{Colorario}[Theorem]                      %Use: \begin{Corollary}[Name]\label{Nombre}...
    \newtheorem{Lemma}[Theorem]{Lemma}                              %Use: \begin{Lemma}[Name]\label{Nombre}...
    \newtheorem{Definition}{Definición}[section]                    %Use: \begin{Definition}[Name]\label{Nombre}...
    \theoremstyle{break}                                            %THEOREMS START 1 SPACE AFTER

    % =====  LOGIC  =====================
    \newcommand \lIff    {\leftrightarrow}                          %Use: \lIff for logic iff
    \newcommand \lEqual  {\MiniSpace \Leftrightarrow \MiniSpace}    %Use: \lEqual for a logic double arrow
    \newcommand \lInfire {\MiniSpace \Rightarrow \MiniSpace}        %Use: \lInfire for a logic infire
    \newcommand \lLongTo {\longrightarrow}                          %Use: \lLongTo for a long arrow

    % =====  FAMOUS SETS  ===============
    \DeclareMathOperator \Naturals     {\mathbb{N}}                 %Use: \Naturals por Notation
    \DeclareMathOperator \Primes       {\mathbb{P}}                 %Use: \Primes por Notation
    \DeclareMathOperator \Integers     {\mathbb{Z}}                 %Use: \Integers por Notation
    \DeclareMathOperator \Racionals    {\mathbb{Q}}                 %Use: \Racionals por Notation
    \DeclareMathOperator \Reals        {\mathbb{R}}                 %Use: \Reals por Notation
    \DeclareMathOperator \Complexs     {\mathbb{C}}                 %Use: \Complex por Notation
    \DeclareMathOperator \GenericField {\mathbb{F}}                 %Use: \GenericField por Notation
    \DeclareMathOperator \VectorSet    {\mathbb{V}}                 %Use: \VectorSet por Notation
    \DeclareMathOperator \SubVectorSet {\mathbb{W}}                 %Use: \SubVectorSet por Notation
    \DeclareMathOperator \Polynomials  {\mathbb{P}}                 %Use: \Polynomials por Notation
    \DeclareMathOperator \VectorSpace  {\VectorSet_{\GenericField}} %Use: \VectorSpace por Notation
    \DeclareMathOperator \LinealTransformation {\mathcal{T}}        %Use: \LinealTransformation for a cool T
    \DeclareMathOperator \LinTrans {\mathcal{T}}                    %Use: \LinTrans for a cool T


    % =====  CONTAINERS   ===============
    \newcommand{\Set}[1]    {\left\{ \; #1 \; \right\}}             %Use: \Set {Info} for INTELLIGENT space 
    \newcommand{\bigSet}[1] {\big\{  \; #1 \; \big\}}               %Use: \bigSet  {Info} for space 
    \newcommand{\BigSet}[1] {\Big\{  \; #1 \; \Big\}}               %Use: \BigSet  {Info} for space 
    \newcommand{\biggSet}[1]{\bigg\{ \; #1 \; \bigg\}}              %Use: \biggSet {Info} for space 
    \newcommand{\BiggSet}[1]{\Bigg\{ \; #1 \; \Bigg\}}              %Use: \BiggSet {Info} for space 
    
    \newcommand{\Brackets}[1]    {\left[ #1 \right]}                %Use: \Brackets {Info} for INTELLIGENT space
    \newcommand{\bigBrackets}[1] {\big[ \; #1 \; \big]}             %Use: \bigBrackets  {Info} for space 
    \newcommand{\BigBrackets}[1] {\Big[ \; #1 \; \Big]}             %Use: \BigBrackets  {Info} for space 
    \newcommand{\biggBrackets}[1]{\bigg[ \; #1 \; \bigg]}           %Use: \biggBrackets {Info} for space 
    \newcommand{\BiggBrackets}[1]{\Bigg[ \; #1 \; \Bigg]}           %Use: \BiggBrackets {Info} for space 
    
    \newcommand{\Wrap}[1]    {\left( #1 \right)}                    %Use: \Wrap {Info} for INTELLIGENT space
    \newcommand{\bigWrap}[1] {\big( \; #1 \; \big)}                 %Use: \bigBrackets  {Info} for space 
    \newcommand{\BigWrap}[1] {\Big( \; #1 \; \Big)}                 %Use: \BigBrackets  {Info} for space 
    \newcommand{\biggWrap}[1]{\bigg( \; #1 \; \bigg)}               %Use: \biggBrackets {Info} for space 
    \newcommand{\BiggWrap}[1]{\Bigg( \; #1 \; \Bigg)}               %Use: \BiggBrackets {Info} for space 
    
    \newcommand{\Generate}[1]{\left\langle #1 \right\rangle}        %Use: \Wrap {Info} for INTELLIGENT space

    % =====  BETTERS MATH COMMANDS   =====
    \newcommand{\pfrac}[2]{\Wrap{\dfrac{#1}{#2}}}                   %Use: Put fractions in parentesis

    % =========================================
    % ====   LINEAL ALGEBRA & VECTORS    ======
    % =========================================

    % ===== UNIT VECTORS  ================
    \newcommand{\hati} {\hat{\imath}}                               %Use: \hati for unit vector    
    \newcommand{\hatj} {\hat{\jmath}}                               %Use: \hatj for unit vector    
    \newcommand{\hatk} {\hat{k}}                                    %Use: \hatk for unit vector

    % ===== FN LINEAL TRANSFORMATION  ====
    \newcommand{\FnLinTrans}[1]{\mathcal{T}\Wrap{#1}}               %Use: \FnLinTrans for a cool T
    \newcommand{\VecLinTrans}[1]{\mathcal{T}\pVector{#1}}           %Use: \LinTrans for a cool T
    \newcommand{\FnLinealTransformation}[1]{\mathcal{T}\Wrap{#1}}   %Use: \FnLinealTransformation

    % ===== MAGNITUDE  ===================
    \newcommand{\abs}[1]{\left\lvert #1 \right\lvert}               %Use: \abs{expression} for |x|
    \newcommand{\Abs}[1]{\left\lVert #1 \right\lVert}               %Use: \Abs{expression} for ||x||
    \newcommand{\Mag}[1]{\left| #1 \right|}                         %Use: \Mag {Info} 
    
    \newcommand{\bVec}[1]{\mathbf{#1}}                              %Use for bold type of vector
    \newcommand{\lVec}[1]{\overrightarrow{#1}}                      %Use for a long arrow over a vector
    \newcommand{\uVec}[1]{\mathbf{\hat{#1}}}                        %Use: Unitary Vector Example: $\uVec{i}

    % ===== ALL FOR DOT PRODUCT  =========
    \makeatletter                                                   %WTF! IS THIS
    \newcommand*\dotP{\mathpalette\dotP@{.5}}                       %Use: \dotP for dot product
    \newcommand*\dotP@[2] {\mathbin {                               %WTF! IS THIS            
        \vcenter{\hbox{\scalebox{#2}{$\m@th#1\bullet$}}}}           %WTF! IS THIS
    }                                                               %WTF! IS THIS
    \makeatother                                                    %WTF! IS THIS

    % === WRAPPERS FOR COLUMN VECTOR ===
    \newcommand{\pVector}[1]                                        %Use: \pVector {Matrix Notation} use parentesis
        { \ensuremath{\begin{pmatrix}#1\end{pmatrix}} }             %Example: \pVector{a\\b\\c} or \pVector{a&b&c} 
    \newcommand{\lVector}[1]                                        %Use: \lVector {Matrix Notation} use a abs 
        { \ensuremath{\begin{vmatrix}#1\end{vmatrix}} }             %Example: \lVector{a\\b\\c} or \lVector{a&b&c} 
    \newcommand{\bVector}[1]                                        %Use: \bVector {Matrix Notation} use a brackets 
        { \ensuremath{\begin{bmatrix}#1\end{bmatrix}} }             %Example: \bVector{a\\b\\c} or \bVector{a&b&c} 
    \newcommand{\Vector}[1]                                         %Use: \Vector {Matrix Notation} no parentesis
        { \ensuremath{\begin{matrix}#1\end{matrix}} }               %Example: \Vector{a\\b\\c} or \Vector{a&b&c}

    % === MAKE MATRIX BETTER  =========
    \makeatletter                                                   %Example: \begin{matrix}[cc|c]
    \renewcommand*\env@matrix[1][*\c@MaxMatrixCols c] {             %WTF! IS THIS
        \hskip -\arraycolsep                                        %WTF! IS THIS
        \let\@ifnextchar\new@ifnextchar                             %WTF! IS THIS
        \array{#1}                                                  %WTF! IS THIS
    }                                                               %WTF! IS THIS
    \makeatother                                                    %WTF! IS THIS

    % =========================================
    % =======   FAMOUS FUNCTIONS   ============
    % =========================================

    % == TRIGONOMETRIC FUNCTIONS  ====
    \newcommand{\Cos}[1] {\cos\Wrap{#1}}                            %Simple wrappers
    \newcommand{\Sin}[1] {\sin\Wrap{#1}}                            %Simple wrappers
    \newcommand{\Tan}[1] {tan\Wrap{#1}}                             %Simple wrappers
    
    \newcommand{\Sec}[1] {sec\Wrap{#1}}                             %Simple wrappers
    \newcommand{\Csc}[1] {csc\Wrap{#1}}                             %Simple wrappers
    \newcommand{\Cot}[1] {cot\Wrap{#1}}                             %Simple wrappers

    % === COMPLEX ANALYSIS TRIG ======
    \newcommand \Cis[1]  {\Cos{#1} + i \Sin{#1}}                    %Use: \Cis for cos(x) + i sin(x)
    \newcommand \pCis[1] {\Wrap{\Cis{#1}}}                          %Use: \pCis for the same with parantesis
    \newcommand \bCis[1] {\Brackets{\Cis{#1}}}                      %Use: \bCis for the same with Brackets


    % =========================================
    % ===========     CALCULUS     ============
    % =========================================

    % ====== TRANSFORMS =============
    \newcommand{\FourierT}[1]{\mathscr{F} \left\{ #1 \right\} }     %Use: \FourierT {Funtion}
    \newcommand{\InvFourierT}[1]{\mathscr{F}^{-1}\left\{#1\right\}} %Use: \InvFourierT {Funtion}

    % ====== DERIVATIVES ============
    \newcommand \MiniDerivate[1][x] {\dfrac{d}{d #1}}               %Use: \MiniDerivate[var] for simple use [var]
    \newcommand \Derivate[2] {\dfrac{d \; #1}{d #2}}                %Use: \Derivate [f(x)][x]
    \newcommand \MiniUpperDerivate[2] {\dfrac{d^{#2}}{d#1^{#2}}}    %Mini Derivate High Orden Derivate -- [x][pow]
    \newcommand \UpperDerivate[3] {\dfrac{d^{#3} \; #1}{d#2^{#3}}}  %Complete High Orden Derivate -- [f(x)][x][pow]
    
    \newcommand \MiniPartial[1][x] {\dfrac{\partial}{\partial #1}}  %Use: \MiniDerivate for simple use [var]
    \newcommand \Partial[2] {\dfrac{\partial \; #1}{\partial #2}}   %Complete Partial Derivate -- [f(x)][x]
    \newcommand \MiniUpperPartial[2]                                %Mini Derivate High Orden Derivate -- [x][pow] 
        {\dfrac{\partial^{#2}}{\partial #1^{#2}}}                   %Mini Derivate High Orden Derivate
    \newcommand \UpperPartial[3]                                    %Complete High Orden Derivate -- [f(x)][x][pow]
        {\dfrac{\partial^{#3} \; #1}{\partial#2^{#3}}}              %Use: \UpperDerivate for simple use

    \DeclareMathOperator \Evaluate  {\Big|}                         %Use: \Evaluate por Notation

    % =========================================
    % ========    GENERAL STYLE     ===========
    % =========================================
    
    % =====  COLORS ==================
    \definecolor{RedMD}{HTML}{F44336}                               %Use: Color :D        
    \definecolor{Red100MD}{HTML}{FFCDD2}                            %Use: Color :D        
    \definecolor{Red200MD}{HTML}{EF9A9A}                            %Use: Color :D        
    \definecolor{Red300MD}{HTML}{E57373}                            %Use: Color :D        
    \definecolor{Red700MD}{HTML}{D32F2F}                            %Use: Color :D 

    \definecolor{PurpleMD}{HTML}{9C27B0}                            %Use: Color :D        
    \definecolor{Purple100MD}{HTML}{E1BEE7}                         %Use: Color :D        
    \definecolor{Purple200MD}{HTML}{EF9A9A}                         %Use: Color :D        
    \definecolor{Purple300MD}{HTML}{BA68C8}                         %Use: Color :D        
    \definecolor{Purple700MD}{HTML}{7B1FA2}                         %Use: Color :D 

    \definecolor{IndigoMD}{HTML}{3F51B5}                            %Use: Color :D        
    \definecolor{Indigo100MD}{HTML}{C5CAE9}                         %Use: Color :D        
    \definecolor{Indigo200MD}{HTML}{9FA8DA}                         %Use: Color :D        
    \definecolor{Indigo300MD}{HTML}{7986CB}                         %Use: Color :D        
    \definecolor{Indigo700MD}{HTML}{303F9F}                         %Use: Color :D 

    \definecolor{BlueMD}{HTML}{2196F3}                              %Use: Color :D        
    \definecolor{Blue100MD}{HTML}{BBDEFB}                           %Use: Color :D        
    \definecolor{Blue200MD}{HTML}{90CAF9}                           %Use: Color :D        
    \definecolor{Blue300MD}{HTML}{64B5F6}                           %Use: Color :D        
    \definecolor{Blue700MD}{HTML}{1976D2}                           %Use: Color :D        
    \definecolor{Blue900MD}{HTML}{0D47A1}                           %Use: Color :D  

    \definecolor{CyanMD}{HTML}{00BCD4}                              %Use: Color :D        
    \definecolor{Cyan100MD}{HTML}{B2EBF2}                           %Use: Color :D        
    \definecolor{Cyan200MD}{HTML}{80DEEA}                           %Use: Color :D        
    \definecolor{Cyan300MD}{HTML}{4DD0E1}                           %Use: Color :D        
    \definecolor{Cyan700MD}{HTML}{0097A7}                           %Use: Color :D        
    \definecolor{Cyan900MD}{HTML}{006064}                           %Use: Color :D 

    \definecolor{TealMD}{HTML}{009688}                              %Use: Color :D        
    \definecolor{Teal100MD}{HTML}{B2DFDB}                           %Use: Color :D        
    \definecolor{Teal200MD}{HTML}{80CBC4}                           %Use: Color :D        
    \definecolor{Teal300MD}{HTML}{4DB6AC}                           %Use: Color :D        
    \definecolor{Teal700MD}{HTML}{00796B}                           %Use: Color :D        
    \definecolor{Teal900MD}{HTML}{004D40}                           %Use: Color :D 

    \definecolor{GreenMD}{HTML}{4CAF50}                             %Use: Color :D        
    \definecolor{Green100MD}{HTML}{C8E6C9}                          %Use: Color :D        
    \definecolor{Green200MD}{HTML}{A5D6A7}                          %Use: Color :D        
    \definecolor{Green300MD}{HTML}{81C784}                          %Use: Color :D        
    \definecolor{Green700MD}{HTML}{388E3C}                          %Use: Color :D        
    \definecolor{Green900MD}{HTML}{1B5E20}                          %Use: Color :D

    \definecolor{AmberMD}{HTML}{FFC107}                             %Use: Color :D        
    \definecolor{Amber100MD}{HTML}{FFECB3}                          %Use: Color :D        
    \definecolor{Amber200MD}{HTML}{FFE082}                          %Use: Color :D        
    \definecolor{Amber300MD}{HTML}{FFD54F}                          %Use: Color :D        
    \definecolor{Amber700MD}{HTML}{FFA000}                          %Use: Color :D        
    \definecolor{Amber900MD}{HTML}{FF6F00}                          %Use: Color :D

    \definecolor{BlueGreyMD}{HTML}{607D8B}                          %Use: Color :D        
    \definecolor{BlueGrey100MD}{HTML}{CFD8DC}                       %Use: Color :D        
    \definecolor{BlueGrey200MD}{HTML}{B0BEC5}                       %Use: Color :D        
    \definecolor{BlueGrey300MD}{HTML}{90A4AE}                       %Use: Color :D        
    \definecolor{BlueGrey700MD}{HTML}{455A64}                       %Use: Color :D        
    \definecolor{BlueGrey900MD}{HTML}{263238}                       %Use: Color :D        

    \definecolor{DeepPurpleMD}{HTML}{673AB7}                        %Use: Color :D

    \newcommand{\Color}[2]{\textcolor{#1}{#2}}                      %Simple color environment
    \newenvironment{ColorText}[1]                                   %Use: \begin{ColorText}
        { \leavevmode\color{#1}\ignorespaces }                      %That's is!

    % =====  CODE EDITOR =============
    \lstdefinestyle{CompilandoStyle} {                              %This is Code Style
        backgroundcolor     = \color{BlueGrey900MD},                %Background Color  
        basicstyle          = \tiny\color{white},                   %Style of text
        commentstyle        = \color{BlueGrey200MD},                %Comment style
        stringstyle         = \color{Green300MD},                   %String style
        keywordstyle        = \color{Blue300MD},                    %keywords style
        numberstyle         = \tiny\color{TealMD},                  %Size of a number
        frame               = shadowbox,                            %Adds a frame around the code
        breakatwhitespace   = true,                                 %Style   
        breaklines          = true,                                 %Style   
        showstringspaces    = false,                                %Hate those spaces                  
        breaklines          = true,                                 %Style                   
        keepspaces          = true,                                 %Style                   
        numbers             = left,                                 %Style                   
        numbersep           = 10pt,                                 %Style 
        xleftmargin         = \parindent,                           %Style 
        tabsize             = 4,                                    %Style
        inputencoding       = utf8/latin1                           %Allow me to use special chars
    }
 
    \lstset{style = CompilandoStyle}                                %Use this style

    % =========================================
    % =======   LALA THINGS ===================
    % =========================================

    %\pgfplotsset{compat=1.13}





    % =========================================
    % =======   ALAN THINGS ===================
    % =========================================





%%\usepackage{minted}
\usepackage{pgfplots}
\usepackage{xcolor}
\usepackage{tablefootnote}
\usepackage[toc,page]{appendix}

\usepackage{algpseudocode}
\usepackage{algorithm}


\addto\captionsspanish{%
	\renewcommand\appendixname{Anexo}
	\renewcommand\appendixpagename{Anexos}
}
\setlength{\headheight}{15pt} 
\renewcommand{\footrulewidth}{0.5pt}
\setlength{\parskip}{0.5em}



\title{Práctica n: Título}
\author{3CM3\\
	ESCOM-IPN}






























\bibliographystyle{IEEEtran}
\begin{document}
    \lstset{inputencoding=utf8/latin1}
	\begin{titlepage}
		\begin{center}
			
			% Upper part of the page. The '~' is needed because \\
			% only works if a paragraph has started.
			
			\noindent
			%%\begin{minipage}{0.5\textwidth}
			%%	\begin{flushleft} \large
			%%		\includegraphics[width=0.3\textwidth]{../ipn.png}
			%%	\end{flushleft}
			%%\end{minipage}%
			%%\begin{minipage}{0.55\textwidth}
			%%	\begin{flushright} \large
			%%		\includegraphics[width=0.7\textwidth]{../escom.png}
			%%	\end{flushright}
			%%\end{minipage}
			
			\textsc{\LARGE Instituto Politécnico Nacional}\\[0.5cm]
			
			\textsc{\Large Escuela Superior de Cómputo}\\[1cm]
			
			\textsc{\Large Redes Computacionales}\\[1cm]
			
			% Title
			
			{ \huge Práctica 1: Archivos de Ordenamiento\\[1cm] }
			
			{ \Large Grupo: 3CM3} \\[1cm]


			
            { \Large Equipo: CompilandoConocimiento.com} \\[1cm]
			
			\noindent
			\begin{minipage}{0.5\textwidth}
				\begin{flushleft} \large
					\emph{Integrantes:}\\
					Morales López Laura Andrea\\
				\end{flushleft}
			\end{minipage}%
			\begin{minipage}{0.5\textwidth}
				\begin{flushright} \large
					\emph{Profesora:} \\
					Nidia Cortez
				\end{flushright}
			\end{minipage}
			
			%%\begin{figure}[H]
			%%    \centering
			%%    \includegraphics[scale=0.06]{../foto.jpg}
			%%\end{figure}
			
			\vfill
		\end{center}
	\end{titlepage}
	\maketitle
	
	
	
    \clearpage
    % =====================================================
	% ========                INDICE              =========
	% =====================================================
	\tableofcontents{}
	\label{sec:Index}

	\clearpage



    \chapter{Teoría}
    \section{Definiciones}

	    \subsection{Dato}
	    	Representación gráfica con significado asignado.
	    \subsection{Información}
	    	Conjunto de datos ordenado.
	    \subsection{Sistema}
	    	Conjunto de elementos que se asocian para lograr un objetivo dentro de un entorno de cierto alcance.
	    \subsection{Software}
	    	Conjunto de programas con objetivo y estructura.
	    \subsection{Base de datos}
	    	\begin{itemize}
	    		\item Representa aspectos del mundo real, a esto se le llama Mini Universo.
	    		\item Datos almacenados que tienen un significado interesante.
	    		\item Diseñar, manipular y construir.
	    	\end{itemize}

	    	

    \section{SGGD}
    Es un programa para crear o manipular bases de datos. Algunos ejemplos son: MySQL, MariaDB, Server,Oracle,Informix, DB2,Sybasem MiniSQL,SQLite,Postgress, etc.

    
    \subsection{Motor Evaluador de consultas}
	    \subsubsection{El compilador}
	    \begin{itemize}
	    	\item Lexico
	    	\item Sintáctico
	    	\item Semántico (Congruencia)
	    \end{itemize}
	    \subsubsection{Optimizador de consultas}
	    \begin{itemize}
	    	\item Basada en costos:
	    			\begin{itemize}
	    				\item Para número de reuniones pequeños es aceptable, para números grandes es dificil de manipular.
	    				\item Se pueden manipular en subconjuntos y calcular individualmente el mejor orden de reunión eliminando los más costosos de cada subconjunto.

	    				Usando esta técnica se puede implementar el algoritmo de programación  dinámica para la optimización del orden de reunión óptimo.
	    				\item Una desventaja es el costo de la misma optimización. El número de los planes de evaluación distintos para una consulta puede ser grande y encontrar el plan optimo lleva mucho trabajo de cómputo.
	    			\end{itemize}
	    	\item Heurística:
	    	\begin{itemize}
	    		\item Suele a ayudar a reducir costo.
	    		\item "Realizar las operaciones de selección tan pronto como sea posible"
	    		\item Se dice que la regla anterior es heurística porque ayuda a reducir el costo, aunque no lo haga siempre.
	    		\item "Realizar las proyecciones tan pronto como sea posible"

	    	\end{itemize}
	    \end{itemize}
    \subsection{Gestión de almacenamiento}
    Depende del sistema operativo.
    \begin{itemize}
    	\item Archivos y métodos de acceso (Búsqueda)
    	\item Administrador de página (Gestiona Caché y RAM)
    	\item Administrador de espacio en disco (RW)
    \end{itemize}

    \subsection{Módulo de control de concurrencia ACID}
    \begin{itemize}
    	\item $A$ tomicidad: Se ejecuta todo o nada.
    	\item $C$ onsistencia: Es el estado coherente de la información o datos que contiene y que se relacionan, en el cual la información cumple las necesidades o expectativas de quien la requiera.
    	\item $I$ Aislamiento: Ejecutar transacciones independientes.
    	\item $D$ urabilidad: Reflejar cambios de la transacción realizada.
    \end{itemize}
    \subsubsection{Administrador de almacenamiento}
    Garantiza que las transacciones soliciten y liberen los bloqueos de acuerdo con el
	correspondiente protocolo de bloqueo y programa la ejecución de las
	transacciones.
	\subsubsection{Administrador de bloqueos}
	Realiza un seguimiento de las solicitudes de bloqueo y concede los bloqueos
	sobre los objetos de la base de datos cuando quedan disponibles.


    \subsection{Administración de tolerancia a fallos}
    Responsable del mantenimiento de un registro y de la restauración del sistema a un estado consistente tras los fallos.


			\begin{figure}[H]
			    \centering
			    \includegraphics[scale=.7]{SGBD.png}
			\end{figure}
		
\section{Clasificación de los SGBD}
	\subsection{Modelo de datos}
	Modelo: Es una respresentación de la realidad que contiene características generales de algo que se va a realizar.

	Modelo de Datos: Conjunto de conceptos que pueden servir para describir la estructura de la Base de Datos. La estructura esta formada por: Datos, Tipo de datos, Vinculos, Restricciones, y puede incluir un conjunto de operaciones para especificar lecturas y actualizaciones.

	Por ejemplo: Jerarquico, Relacional, Red, Orientado a Objetos, Entidad-Relación.
	\subsection{Número de Usuarios}
	\begin{itemize}
		\item Monousuario
		\item Multiusuario
	\end{itemize}
	\subsection{Proposito}
	\begin{itemize}
		\item Específico
		\item General
	\end{itemize}
	\subsection{Costos}
	\begin{itemize}
		\item Gratis
		\item Licencia
	\end{itemize}
	\subsection{Número de sitios}
	\begin{itemize}
		\item Centralizado
		\item Distribuido: En varios servidores para evitar fallos.
		\begin{itemize}
			\item Homogeneo: Los servicios son del mismo control
			\item Heterogeneo: Interfaces de comunicación
		\end{itemize}
	\end{itemize}
\section{Clasificación de Modelo de Datos}
\begin{itemize}
	\item Alto nivel/Conceptual Tiene conceptos muy cercano al modo en que el usuario percibe los datos.
	\item Representación/Implementación Sus conceptos pueden ser entendidos  por los usuarios finales aunque no estan alejados de la forma en que estan organizados en la Base de Datos.
	\item Bajo nivel/Fisico Describe los detalles de como se almacenan los datos.
\end{itemize}

\section{Elementos de un Modelo de Datos}
\begin{itemize}
	\item Entidad: Representa un objeto del mundo Real
	\item Atributo: Característica que permite descubrir una entidad.
	\item Vinculo: Describe una interacción entre dos entidades.
	\item Esquema de la Base de Datos: Descripción de la estructura de la Base de Datos.
	\item Diagrama del esquema de la Base de Datos: Representación gráfica del esquema de la base de datos.
	\item Estado de la Base de Datos: Datos en determinado tiempo.
\end{itemize}

%%%%%%%%%%%%%%%%%%%%%%%%%%%%%%%%%%%%%%%%%%%%%%%%%%%%%%%%%%%%%%%%%%%%%%%%%




    Administrador de transacciones(estrategia de ejecuación)

    Administrador de bloqueos (Granularidad)

    Bitácora.






    \subsection{Arquitectura de un SGBD}

		\subsubsection{Tipos de usuarios}
		\begin{itemize}
			\item Usuarios no sofisticados: 
			\item Usuarios sofsticados:
		\end{itemize}
		\subsubsection{Formas Web}
		\subsubsection{Aplicaciónes front-end}
		\subsubsection{Intefaces SQL}

	\subsection{SQL Lenguaje Estructurado de Consultas}

		\subsubsection{Bloque 1}
		\begin{itemize}
			\item Administrador de transacciones
			\item Administrador de bloques
		\end{itemize}

		\subsubsection{Bloque 2}
		\begin{itemize}
			\item Ejecutor del Plan
			\item PARSER
			\item Evaluador de Operaciones
			\item OPTIMIZADOR
		\end{itemize}

		\subsubsection{Bloque 3}
		\begin{itemize}
			\item Archivos y métodos de acceso
			\item Administrador de páginas
			\item Administrador de Espacio en disco
		\end{itemize}

		\subsubsection{Bloque 4}
		\begin{itemize}
			\item Administrador de Recuperación
			
		\end{itemize}

		\subsubsection{Bloque 5 Base}

		Catalogo de Sistema
		\begin{itemize}
			\item Archivo de datos.
			\item Archivos de indices.
		\end{itemize}

		%%Imagen


	\subsection{Personas involucradas en un SBD}

		\subsubsection{Administrador de Base de Datos}
		\subsubsection{Diseñador de BD}
		\subsubsection{Analistas de Sistemas-Programadores}
		\subsubsection{Usuarios Finales}
		\begin{itemize}
			\item Ususarios Esporádicos
			\item Usuarios paramétricos/Simples
			\item Usuarios Autónomos
			\item Usuarios Avanzados
		\end{itemize}

    \section{Normalización}
    \subsection{Que es?}

    "Es el proceso de eliminación de redundancias en una tabla para que sea más fácil"
    \subsection{Descripcion}
    Tenemos que tener una relación bien estructurada, osea que contenga el minimo de redundancia y permita a los usuarios insertar, modificar y borrar registros en una tabla sin errores o inconsistencias.

    Ademas debemos de evitar los 3 tipos de anomalías: Anomalías de inserción, anomalías en eliminación y anomalías en actualización.

    Aqui es donde entra la normalización, el cual es un proceso formal ara decidir que atributos deberían ser agrupados en una relación, es el proceso de descomponer relaciones con anomalías para producir relaciones pequeñas y bien estructuradas.
    \subsection{Dependencias Funcionales}
    Es una restricción entre dos conjuntos de atributos de
la base de datos.
Una dependencia funcional, denotada por $X \rightarrow Y$, entre dos conjuntos de
atributos X e Y que son subconjuntos de R, especifica una restricción sobre las
posibles tuplas que podrán formar un estado de relación r de R.


La restricción dice que, para dos tuplas cualesquiera t 1 y t 2 , de r tales que t 1 [X] =
t 2 [X], debemos tener también t 1 [Y] = t 2 [Y]. Esto significa que los valores del
componente Y de una tupla r dependen de los valores del componente X, o están
determinados por ellos; o bien, que los valores del componente X de una tupla
determinan la manera única (o funcionalmente) los valores del componente Y.
    \subsection{Forma Normal}
    Es un estado de una relación que resulta de aplicar simples
reglas tomando en cuenta la dependencia funcional (o relaciones entre los
atributos) de una relación.


    \begin{itemize}
        \item 1. Primera Forma Normal.(Redundancia y Atributos atómicos)
Cualquier atributo multivalor (también llamado grupo repetitivo) tiene que ser
eliminado.

Eliminar redundancia y convertir los
atributos complejos en atributos atómicos, o no descomponibles.


\item 2. Segunda Forma Normal(Los atributos solo depende de una clave).
Cualquier dependendencia funcional parcial tienen que ser eliminadas, es decir,
los atributos no claves son identificados por la llave primaria.

Un esquema de relación R está en 2FN si todo atributo no primo A en R depende
funcionalmente de manera total de la clave primaria de R.

Se le puede “normalizar en 2FN"
dando lugar a varias relaciones 2FN en las que los atributos no primos estén
asociados sólo a la parte de la clave primaria de la que dependen funcionalmente de manera total.


\item 3. Tercera Forma Normal(Atributos no llave no dependen de atributos no llave).
Cualquier dependendencia funcional parcial tienen que ser eliminadas, es decir,
los atributos no claves son identificados por la llave primaria.

Se elimina las dependencias transitivas; es decir atributos no clave no
dependen de otros atributos no clave.
\item 4. Boyce/Codd Forma Normal(Claves no dependan de atributos no cklave).
Cualquier anomalía resultante de dependendencias funcionales tienen que ser
eliminadas.

Un atributo clave  es funcionalmente dependiente de un atributo no clave.

“Una relación R está en la FNBC si y sólo si cada determinante es una llave
candidata".

La relación es modificada de tal manera que el determinante de la relación que
no es una llave candidata llega ser un componente de la llave primaria de la
relación revisada. El atributo que es funcionalmente dependiente en el
determinante llega a ser un atributo no clave. Esto es válido por la dependencia
funcional
\item 5. Cuarta Forma Normal.
Cualquier dependendencia multivaluada tienen que ser eliminadas.
\item 6. Quinta Forma Normal.
Cualquier dependendencia de junta o de proyección tienen que ser eliminadas.
    \end{itemize}



























\chapter{SQL}
	\section{Introducción}
	Es un sublenuaje formado por:
	\begin{itemize}
		\item DDL Lenguaje de definición de datos.
		\item DML Lenguaje de manejo de datos
		\item CTL Lenguaje de control de transacciones.
		\item SDL Lenguaje de definición de almacenamiento.
		\item VDL Lenguaje de definición de vistas.
	\end{itemize}
    \section{Aldunas cosas a tener en cuenta}
    Crea primero todas las restricciones de llaves foraneas y primarias.
    
	\section{Algunas sentencias}

	\emph{Show Databases;} Muestra las bases que tenemos.


	\emph{drop database \textbf{NomBD};} Elimina una base de datos.

	\emph{create database \textbf{NomBD};} Crea una base de datos.

	\emph{use \textbf{NomBD};} Entra a la base de datos.

	\emph{Show tables}Muestra las tables en mi base de datos.
 
	\emph{Create table \textbf{Nomtable}(Atributos);} Crear una tabla donde atributo tiene nombre de variable, tipo y restricción.

	\emph{describe \textbf{Nomtable};} Muestra la tabla.

	\emph{show create table \textbf{Nomtable};} Describe la manera de creacion de la tabla.

	\emph{Alter table \textbf{Nomtable} add constraint PK Primary Key (idP);} Modificar una tabla y agrega restricción llamada Pk(unico) con su tipo Primary key Nombre del atributo.

	\emph{Alter table \textbf{Nomtable} add constraint FKP Foreign Key (\textbf{NomDato}) references \textbf{Nomtable}(\textbf{NomDato});}Ligar tablas. Modificamos la table una restriccion de llave foranea (cual) una referencia a tabla (Dato)


	\emph{Insert into \textbf{NomTable}(Atributos) values(Valores Atributos);} Inserta valores en una tabla respecto a su posición.

	\emph{Select * From \textbf{NomTable};} Muestra una tabla.

	\emph{alter table NomTable add column NomDato type null;}Agregar una columna

	\emph{alter table NomTable modify NomDato TypetoChange null;} Modificar el tipo de dato de una columna.

	\emph{alter table NomTable change NomDato NewName Type null;} Cambiar Nombre de dato;

	\emph{alter table NomTable drop column NomDato;} Eliminar una columna de una tabla.

	\emph{Delete from nombre;} Eliminar tabla;


	\emph{Delete from NomTable where IdP="P3";} Eliminar un dato.


	\emph{alter table NomTable drop Foreign Key nombreConstraint;}

	\emph{Delete from NomTable where IdP="P2"} Borra en cascada, si en la tabla original se desea eliminar entonces en la dependiente tambien se elimina


	\emph{Alter table \textbf{Nomtable} add constraint FKP Foreign Key (\textbf{NomDato}) references \textbf{Nomtable}(\textbf{NomDato}) on delete cascade on update cascade;} Ligar tablas. Modificamos la table una restriccion de llave foranea (cual) una referencia a tabla (Dato) y decimos que elimine en cascada y actualizacion tambian.

	\emph{update NomTable set NomAtrib="P7" where NomAtrib="P3"}Modificarel valor de un dato ya contenido en la tabla en cascada en un tipo de atributo.


	   \emph{Select * from \textbf{NomTable} order by \textbf{NomAtrib} ASC;} Debe ir al final de la sentencia y ASC es orden ascendente aunque este esta por default

    \emph{Select * from \textbf{NomTable} order by \textbf{NomAtrib} desc;} Debe ir al final de la sentencia y desc es orden descendente

        \emph{Select * from \textbf{NomTable} order by \textbf{NomAtrib1} desc, \textbf{NomAtrib2} asc;} Ordena primero por Atrib1 y despues por Atrib2

    \emph{Select distinct  \textbf{NomAtrib} from \textbf{Nomtable};} Elimina duplicados.

    \emph{Select * from \textbf{NomTable},(\emph{Select * from \textbf{NomTable}}) A;} Regresa un producto cruz  y se crea una relación A que es la parte donde esta el parentesis, no se guarda fisicamente..


      \emph{Select count(\textbf{NomAtrib}) from \textbf{NomTable};} Cuenta tuplas sin contar nulls.


      \emph{Select sum(\textbf{NomAtrib}) from \textbf{NomTable};} Suma.

      \emph{Select avg(\textbf{NomAtrib}) from \textbf{NomTable};} Promedio.

      \emph{Select min(\textbf{NomAtrib}) from \textbf{NomTable};} Valor minimo. En alfanuméricos es por orden de alfabeto.

      \emph{Select max(\textbf{NomAtrib}) from \textbf{NomTable};} Valor maximo. En alfanuméricos es por orden de alfabeto.


      Suma y promedio solo para datos numericos.
      max min y contador para alfanumericos

      \emph{Select (\textbf{NomAtrib}) from \textbf{NomTable} group by \textbf{NomAtrib2};} Agrupar por atrib2

    \subsection{Crear un respaldo de la BD}
    \emph{Mysqldump } \textbf{- -user=root} \text{- -password=root} \emph{nombrebd}>Dir/archivo.sql
    Donde dir es la direccion donde se guardará el archivo resultado.

    \subsection{Crear un repado por un puerto diferente}
    \emph{Mysqldump } \textbf{- -user=root} \text{-P 3309} \emph{nombrebd}>Dir/archivo.sql
    Donde -P es el puerto a escuchar.
    \subsection{Crear una bd a partir de un respaldo}
    \emph{Mysql} \textbf{- -user=root} \text{- -password=root}  \text{-P 3309} \emph{nombrebd}<Dir/archivo.sql


    


    mysql --user=root --password=root Parejas2<BD2CM1/archivo.sql
    mysqldump --user=root --password=root Parejas>'BD2CM1/archivo.sql'



     mysqldump --user=Lalaandrea10 --password=root Parejas>BD2CM1/archivo.sql


     select nombre as Name from profesor; Alias de nombre es name;

     select (edad+20)x10 as operacion from profesor; Crea una tabla llamada (edad+20)x10 y la renombra como operación




	\begin{appendices}
	
	
	\end{appendices}
	
    \bibliography{../referencias.bib}
	
\end{document}
	